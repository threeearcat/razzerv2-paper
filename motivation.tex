\section{Motivation}
\label{s:motivation}

% - characteristics of concurrency bugs
% - challenges / design requirements
% - limitations of existing approaches

This work is motivated by an observation that none of previous work
properly addresses characteristics of \textit{offending thread
  interleaving} (\ie, one that causes a concurrency bug).
%
As a consequence, previous work either \textbf{1)} suffers from
identifying whether interesting thread interleavings remain
untested~\cite{krace, conzzer, muzz}\yj{use clearer sentence}, or \textbf{2)} waste the
computing power by ineffectively exploring the search space of thread
interleaving~\cite{snowboard, razzer}.



In this section, we first study how concurrency bugs manifest
depending on thread interleaving through a real-world concurrency bug
example.
%
From the study, we elicit design goals to effectively discover concurrency
bugs in the kernel, and discuss why existing approaches have limitations 
in satisfying the design goals.


\PP{Manifestation of concurrency bugs}
%
\begin{figure}[t]
  \centering
  \includegraphics[width=0.9\linewidth]{fig/cve-2017-10661.pdf}
  \caption{Simplified code snippet of CVE-2017-17712. If \texttt{B1}
    is executed between \texttt{A2} and \texttt{A4}, concurrenct
    accesses on \texttt{inet->hdrincl} leads to uninitialized stack
    pointer usage on \texttt{rfv}, and an attacker may gain root
    privileges through a dedicated attack
    technique~\cite{stackspray}.}
  \label{fig:cve-2017-17712}
\end{figure}
%
In \autoref{fig:cve-2017-17712}, an uninitialized access bug may
manifest when two system calls are executed concurrently:
\texttt{sendmsg()} to send a message through an ipv4 socket, and
\texttt{setsockopt()} to modify an option of the ipv4 socket.

Let us assume \texttt{inet->hdrincl} is initially \texttt{1}.
%
During sending a message through the ipv4 socket, thread~A reads a
value of \texttt{inet->hdrincl} twice at \texttt{A2} and \texttt{A4}.
%
However, since these two read operations are not atomically executed,
thread~B may intervene in the middle of these two read operations.
%
In that case, if \texttt{B1} is executed between \texttt{A2} and
\texttt{A4}, thread~A reads different values of \texttt{inet->hdrincl}
at \texttt{A2} and \texttt{A4}, and dereference \texttt{rfv} without
initializing it.


\PP{Observation 1: Combined interleaving orders}
%
This example demonstrates that a concurrency bug is caused by \textit{a combined
  result of multiple interleaving orders}, where each interleaving order
denotes the execution order between an instruction pair that access
the same memory object.
%
In the example of \autoref{fig:cve-2017-17712}, two interleaving orders
are combined to eventually cause the uninitialized access bug.
%
First, \texttt{A2} should be executed before \texttt{B1} (\ie,
$\texttt{A2} \Rightarrow \texttt{B1}$\footnote{In this paper,
  $\texttt{X} \Rightarrow \texttt{Y}$ denotes that \texttt{X} is
  executed before \texttt{Y}}) to make thread~A not initialize
\texttt{rfv}.
%
Second, \texttt{B1} should be executed before \texttt{A4} (\ie,
$\texttt{B1} \Rightarrow \texttt{A4}$) to make thread~A dereference
uninitialized \texttt{rfv} while other interleavings do not 
contribute to the bug.
%
%Therefore, thread interleavings satisfying a combination of the two
%interleaving orders (\ie,
%$(\texttt{A2} \Rightarrow \texttt{B1}) \wedge (\texttt{B1} \Rightarrow
%\texttt{A4}))$ cause the uninitialized access bug, while all other
%thread interleavings do not.
Therefore, to trigger kernel concurrency bugs, a fuzzer must 
consider a combination of multiple interleaving orders. e.g.,
$(\texttt{A2} \Rightarrow \texttt{B1}) \wedge (\texttt{B1} \Rightarrow
\texttt{A4}))$ in \autoref{fig:cve-2017-17712}.


\PP{Design goal 1: Informative interleaving coverage}
\yj{Revisit text to get the point across better}
The observation gives an insight of how to define coverage metric 
in a concurrency fuzzer.
To discover the uninitialized access bug, 
interleaving coverage should not be saturated until
$(\texttt{A2} \Rightarrow \texttt{B1}) \wedge (\texttt{B1} \Rightarrow
\texttt{A4})$ is executed.
%
Otherwise, a fuzzer may think that there is no more interesting thread
interleaving in the multi-thread input, and stop searching for new
thread interleavings of the multi-thread input, missing the
uninitialized access bug.


\PP{Observation 2: Feedback from explored executions}
%
An explored execution that did not cause a concurrency bug provides useful feedback 
to guide what interleaving orders should be further explored.

In the example of \autoref{fig:cve-2017-17712}, let us assume 
if a fuzzer executes the two system calls \textit{sequentially} 
such that thread~A executes all instructions followed by the execution of thread~B.
From the sequential execution, a fuzzer observes that the three instructions,
\texttt{A2}, \texttt{A4}, and \texttt{B1}, are \yj{can a fuzzer know they are conflicting instructions?} executed in the order of
$\texttt{A2} \Rightarrow \texttt{A4} \Rightarrow \texttt{B1}$. 
The execution order does not cause the uninitialized access bug.
%
However, from the explored execution, one can easily imagine a new interleaving order of these three
instructions by changing the execution order of \texttt{A4} and
\texttt{B1} (\ie,
$\texttt{A2} \Rightarrow \texttt{B1} \Rightarrow \texttt{A4}$).
%
The \yj{speculative} interleaving is what exactly we are looking for; it
satisfies the combination of interleaving orders
$(\texttt{A2} \Rightarrow \texttt{B1}) \wedge (\texttt{B1} \Rightarrow
\texttt{A4}))$, and, if executed, the interleaving order triggers 
the uninitialized access bug.

\PP{Design goal 2: Coverage-based interleaving search strategy}
The observation gives a direction that how a fuzzer should explore 
the search space using a coverage metric.
If an interleaving coverage metric tracks\yj{metric tracks?} that
$\texttt{A2} \Rightarrow \texttt{A4} \Rightarrow \texttt{B1}$ is
\textit{explored} before, a fuzzer can utilize interleaving
coverage as feedback to infer \textit{unexplored} thread interleavings (\eg,
$\texttt{A2} \Rightarrow \texttt{B1} \Rightarrow \texttt{A4}$).
%
In this way, a fuzzer can effectively determine what to execute in
future iterations, and quickly discover the uninitialized access bug
without redundantly executing thread interleavings.



\subsection{Limitation of prior approaches}
\label{ss:existingapproaches}
\begin{table}[t]
  \centering
  \resizebox{\linewidth}{!}{
  \begin{tabular}{l l l}
    \toprule
     & \thead{\textbf{Concurrency} \\ \textbf{coverage metric (R1)}} & \thead{\textbf{Thread scheduling} \\ \textbf{control mechanism (R2)}} \\
    \midrule
    \textbf{Razzer~\cite{razzer}} & -- & XXX \\
    \textbf{KRace~\cite{krace}} & Alias coverage & Delay injection \\
    \textbf{Snowboard~\cite{snowboard}} & -- & XXX \\
    \bottomrule
  \end{tabular}
}

%%% Local Variables:
%%% mode: latex
%%% TeX-master: "../p"
%%% End:

  \caption{Interleaving coverage metrics and interleaving search
    strategy of recent concurrency fuzzing. ``--'' indicates that a
    fuzzer does not adopt a concurrency coverage metric. \dr{TODO:
      rewording}}
  \label{table:motivation}
\end{table}

\autoref{table:motivation} summarizes interleaving coverage metrics and thread scheduling control mechanisms of prior approaches.
%
Even though they achieve their own successes, we find that
their interleaving coverage metrics and thread scheduling control
mechanisms do not satisfy \textbf{Design goal 1} and \textbf{2}.


\PP{Less-informative interleaving coverage}
%
We find that previously proposed interleaving coverage metrics are
less-informative to determine whether a multi-thread input needs to be
further tested. Thus, they do not satisfy the \textbf{Design goal 1}.
%
This is because none of them consider a combination of interleaving
orders. 
%
Specifically, alias coverage tracks individual interleaving orders,
and suffer from describing a combination of interleaving orders.
%
On the other hand, concurrenct call pair trakcs a pair of
concurrently-executed functions, and do not differentiate
interleavings taken place in the same function pair.




\begin{figure}[t]
  \centering
  \includegraphics[width=0.95\linewidth]{fig/alias-coverage.pdf}
  \caption{Two thread interleavings between thread~A and thread~B
    described in \autoref{fig:cve-2017-17712}. The uninitialized
    access bug does not manifest in both interleavings. We
    intentionally omits bug-irrelevant memory accesses (\ie,
    \texttt{A6} and \texttt{B2}).}
  \label{fig:alias-coverage}
\end{figure}


\autoref{fig:alias-coverage} shows two interleavings that saturate
existing interleaving coverage, where the uninitialized access bug
does not manifest.
%
In the two interleavings described in this figure, we can observe all
four individual interleaving orders regarding \texttt{inet->hdrincl}
(\ie, $\texttt{A2} \Rightarrow \texttt{B1}$ and
$\texttt{A4} \Rightarrow \texttt{B1}$ in Interleaving~\#1,
$\texttt{B1} \Rightarrow \texttt{A2}$, and
$\texttt{B1} \Rightarrow \texttt{A4}$ in Interleaving~\#2).
%
Therefore, alias coverage is saturated with these two thread
interleavings.
%
Moreover, concurrent call pair is also saturated with these two thread
interleavings, because once two functions \texttt{sendmsg()} and
\texttt{setsockop()} are executed concurrently, concurrnet call pair
does not matter how a thread interleaving occurs within the function
pair.

Therefore, they can possibly be saturated even before a concurrency
bug is exposed, misleading a fuzzer to de-prioritize a multi-thread
input in which a concurrency bug resides.




% \yj{This paragraph must be easy enough for reader to intuitively understand, but hard to digest discussions}
% However, they are not applicable to track behavioral changes\yj{what does it mean?} according
% to a combination of interleaving orders, mainly because they either
% track only a \textit{single} interleaving order~\cite{krace, muzz} or
% \textit{coarse-grained information} such as a pair of two
% concurrently-executed functions~\cite{conzzer}.
% \yj{I do not understand why tracking a single int. order and coarse-grain information are not able to track behavioral changes}

% Taking the example of KRace's alias coverage,
% \autoref{fig:alias-coverage} describes two thread interleavings that
% saturate alias coverage found between two system calls in
% \autoref{fig:cve-2017-17712}.
% %
% In these example interleaving scenarios, the uninitialized access does
% not manifest even after alias coverage is saturated, and a fuzzer may
% decide to stop searching for new thread interleavings in the two
% system calls.
% %
% While we do not enumerate all proposed interleaving coverage metrics
% here, we find that they all share the same limitation.


\PP{Coverage-oblivious interleaving search strategy}
%
Stemming from less-informative coverage metrics, proposed interleaving
search strategies are not properly directed by interleaving coverage,
and thus, do not satisfy the \textbf{Design goal 2}.

Since Razzer~\cite{razzer} and Snowboard~\cite{snowboard} do not make
use of interleaving coverage at all, they cannot rely on previous
executions in inferring unexplored thread interleavings.
%
Krace~\cite{krace} explores thread interleavings randomly without
considering what thread interleavings are explored before.

Whereas, Conzzer~\cite{conzzer} is the only work that attempts to
direct a fuzzer based on interleaving coverage.
%
However, its interleaving coverage (\ie, concurrent call pair) tracks
thread interleavings in the function-level granularity, and is limited
in distinguishing tested interleavings and untested interleavings in
the instruction-level granularity.
%
In other words, the Conzzer's interleaving search strategy can direct
a fuzzer to execute two functions \texttt{raw_sendmsg()} and
\texttt{do_ip_setsockopt()} concurrently.
%
But even after that, Conzzer suffers from triggering the uninitialized
bug since it is not aware of the execution order of instructions.


\dr{TODO: MUZZ}


%%% Local Variables:
%%% mode: latex
%%% TeX-master: "p"
%%% End:
