\section{Conclusion}
\label{s:conclusion}

% We proposed RAZZER, a fuzz testing tool tailored to find race
% bugs. It utilizes a static analysis to spot potential data race points
% to guide the fuzzer to identify races. Moreover, it modifies the
% underlying hypervisor to trigger a race deterministically. The
% evaluation of RAZZER demonstrates its strong capability to
% detect races. It has thus far detected 30 new races in the Linux
% kernel, and a comparison study with other state-of-the-art tools,
% specifically Syzkaller and SKI, demonstrates its outstanding
% efficiency to detect race bugs in the kernel.

% We perform in-depth studies of Linux kernel concurrency
% bugs and drive three requirements: comprehensiveness, pattern-
% agnostic, and conciseness. This work proposes AITIA to sat-
% isfy the three requirements. AITIA fully automates the process
% of identifying the root cause of kernel concurrency bugs re-
% ported from existing bug-finding systems and analyzes the
% root cause as a form of a causality chain. We evaluate AITIA
% on 22 real-world concurrency bugs and successfully diagnose
% six unfixed bugs using AITIA.


In this paper, we propose \sys, a novel and effective kernel
concurrency fuzzer.
%
The key improvement of \sys resides in 1) adopting informative
interleaving coverage called interleaving segment coverage, and 2)
speculative interleaving exploration by utilizing explored
interleaving coverage.
%
As a result, \sys has discovered \totalbugs previously-undisclosed
concurrency bugs, and a comparison study demonstrates that \sys
outperforms state-of-the-art concurrency fuzzers, showing the
effectiveness of \sys.



%%% Local Variables:
%%% mode: latex
%%% TeX-master: "p"
%%% End:
