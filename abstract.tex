\begin{abstract}
%
Discovering kernel concurrency bugs through fuzzing is challenging.
%
Identifying kernel concurrency bugs, as opposed to non-concurrency
bugs, necessitates an analysis of possible interleavings between two
or more threads. However, because the search space of thread
interleaving is vast, it is impractical to investigate all conceivable
thread interleavings.
%
To explore the vast search space, most previous approaches perform
random or simple heuristic searches without having coverage for thread
interleaving or with an insufficient form of coverage. As a result,
they either conduct wasteful searches with redundant executions or
overlook concurrent bugs that their coverage cannot address.

To overcome such limitations, we propose \sys, a fuzzing framework for
kernel concurrency bugs.
%
When exploring the search space of thread interleavings, \sys
decomposes an entire thread interleaving into a set of segments, each
of which represents an interleaving of the small number of
instructions, and utilizes individual segments as interleaving
coverage, called interleaving segment coverage.
%
When searching for thread interleavings, \sys mutates interleavings in
explored interleaving segments to construct new thread interleavings
that has not yet been explored.
%
With \sys, we discover new 21 concurrency bugs in Linux kernels, and
demonstrate the efficiency of \sys by showing that \sys can identify
known bugs on average 4.1 times quickly than the state-of-the-art
approaches, Snowboard and KRACE.

\end{abstract}

%%% Local Variables:
%%% mode: latex
%%% TeX-master: "p"
%%% End:
