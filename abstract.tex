\begin{abstract}
%
Discovering kernel concurrency bugs through fuzzing is challenging.
%
Identifying kernel concurrency bugs, as opposed to non-concurrency
bugs, necessitates an analysis of possible interleavings between two
or more threads. However, because the search space of thread
interleaving is vast, it is practically impossible to investigate all
conceivable thread interleavings.
%
\red{To explore the vast search space, most previous approaches
  perform simple heuristic searches without having coverage for thread
  interleaving or with an insufficient form of coverage.} As a result,
they either conduct wasteful searches with redundant executions or
overlook concurrent bugs that their coverage cannot address.

To overcome such limitations, we propose \sys, a fuzzing framework for
kernel concurrency bugs.
%
When exploring the search space of thread interleavings, \sys
decomposes entire thread interleavings to a set of segments each of
which contains the small number of instructions, and utilizes
individual segments as interleaving coverage, called \intcov.
%
\red{At each search step, \sys mutates interleavings in each segment to
construct new thread interleavings that has not yet been explored
(\ie, not recorded in the \intcov).}
%
With \sys, we discover new 21 concurrency bugs in Linux kernels and
demonstrate the efficiency by showing that \sys can identify known
bugs up to \yj{xx} times quickly than the state-of-the-art approaches,
Snowboard and KRACE.

\end{abstract}

%%% Local Variables:
%%% mode: latex
%%% TeX-master: "p"
%%% End:
