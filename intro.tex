\section{Introduction}
\label{s:intro}

%\dr{I make a decision to denote ``(thread) interleaving'' as a whole
% interleaving. MUZZ and Krace use the term like this.}

Concurrency bugs are common in modern kernels due to 
the pervasive adoption of efficient but difficult-to-reason
parallelization techniques.
The consequence of kernel concurrency bugs is disastrous. They crash
the entire system, breaking availability~\cite{cve201812232, snowboardbug} or causing data loss~\cite{dataloss}.
Even worse, attackers 
exploit concurrency bugs to mount a privilege escalation attack.
A recent study~\cite{exprace} demonstrates that a user-level attacker 
reliably exploits non-deterministic concurrency bugs without performing
brute-force attacks.

However, discovering kernel concurrency bugs is much more difficult than 
finding non-concurrency bugs. 
In contrast to non-concurrency bugs which are discovered by 
sequential testing of single-thread execution,
kernel concurrency bugs are typically caused by the concurrent execution 
of two or more threads. 
Kernel concurrency bugs only emerge in a particular pattern of 
interleavings between threads, which happens only when a certain timing condition is met.
%
Therefore, to discover kernel concurrency bugs, a fuzzer must consider
two aspects: how to explore execution paths (\ie, sequential aspects),
and how to navigate thread interleavings (\ie, concurrent aspects).

% generate multi-thread
% input seeds (\eg, system call pairs and their parameter values).
%


% Unfortunately, exploring all the possible interleavings is not
% practical because the number of instructions executed by a system call
% in Linux is huge.

%\dr{as far as I know, there are conzzer and muzz}
%\dr{If we consider Snowboard as a fuzzer, Snowboard does not
%  adopt an interleaving coverage metric, so it cannot distinguish two
%  interleavings, and it relies on *heuristics* (= clustering
%  strategies) to determine which interleavings should be explored}

%To improve the efficiency of concurrency fuzzing, several approaches
%have been proposed.  
%Razzer~\cite{razzer} uses static analysis whereas
%Snowboard~\cite{snowboard} uses dynamic analysis
%to identify instruction pairs that access the same objects.
%From detected instruction pairs, they use heuristics 
%to produce new thread interleaving. Alas, they are coverage-oblivious, 
%so they cannot prune redundant executions of thread interleaving
%that produce identical results. We underline that 
%coverage-guided searching is the preferred strategy 
%for concurrency fuzzing. However, what form of coverage would be 
%effective in concurrency fuzzing has not been researched extensively.
%KRACE~\cite{krace} proposes alias coverage, which simply tracks 
%a single alias-pair instruction. We observe that tracking single 
%instruction pairs are not sufficient. Furthermore, KRACE uses 
%the alias coverage to decide whether to continue to explore or not 
%in the current multi-threaded input. When exploring a thread 
%interleaving space, it takes a randomized approach, 
%sharing the similar inefficiencies of previous techniques.


For decades, coverage-guided fuzzing has been studied to effectively
explore the search space of execution paths~\cite{janus, hydra,
  healer, imf, hfl, syzkaller, klee, s2e, kafl}.
%
% proven to be effective in exploring such a huge search space\yj{cite}.
However, little progress has been made in exploring thread
interleavings.
%
Specifically, little is known about how to define the coverage metric
for \textit{thread interleavings} (shortly, interleaving coverage
metric), and how to utilize interleaving coverage metric in exploring
thread interleavings.
%
For example, Razzer~\cite{razzer} and Snowboard~\cite{snowboard} use
own heuristics combined by static or dynamic analysis, but they are
coverage-oblivious.  They cannot prune redundant executions of thread
interleavings that produce identical results.
%
KRACE~\cite{krace} proposes alias coverage, which tracks
concurrently-executed \textit{two} instructions, but alias coverage
suffers from effectively summarizing the behavior of concurrency bugs,
because most concurrency bugs are caused by the execution order of
\textit{more} instructions~\cite{learningfrommistakes}.  Furthermore,
KRACE randomly schedules instructions during runtime without
considering what thread interleavings are explored before.

To overcome the shortcoming of existing techniques, this paper proposes
\sys, a fuzzing framework for kernel concurrency bugs. 
% \sys discovers not only data races but 
% all interleaving-related bugs. 
At its core, \sys aims to systematically explore the vast search space of thread
interleavings to quickly identify
interleavings leading to concurrency bugs. The central idea behind \sys
is i) defining a new, effective interleaving coverage metric that reflects the 
common nature of concurrency bugs, and ii) using that metric to drive 
an efficient search strategy for discovering previously untested thread interleavings.
Hence, we demonstrate that \sys saves a significant amount of CPU expenses 
in kernel fuzzing by finding concurrent bugs quickly.
% without sacrificing bug-detecting capability or even better 
% capability than earlier work.
\sys achieves the goal with the following two key ideas:
segmentizing thread interleaving and the mutation-based thread interleaving exploration.

\PP{Segmentizing thread interleaving}
%
Our key idea starts from the finding of a previous
study~\cite{learningfrommistakes}, stating that most concurrency bugs
occur by a specific execution order of at most four memory accesses to
shared memory objects.
%
Based on this finding, \sys decomposes the large thread interleaving
space into tractable small sub-spaces, called \textit{interleaving
  segment}. The number of instructions in an interleaving segment is
small (at most four by default), and each interleaving segment
represents an interleaving of those instructions.

Using interleaving segments, we propose our novel interleaving
coverage metric, \textit{\intcov}, which is defined as a set of
detected interleaving segments.
%
% \dr{I don't think this is wrong, but we don't have claims or evidences about it}
% \Intcov generalizes coverage 
% metrics proposed by previous approaches where they only track 
% a single interleaving (the size of a segment is 2).
Using \intcov provides two benefits to fuzzers. First, a fuzzer can
take into account interactions of multiple instructions, which
increases accuracy of discovering concurrency bugs in comparison to
previous approaches~\cite{krace, conzzer}.  Second, using \intcov, a
fuzzer can produce thread interleavings that have not yet been found
for the subsequent search.

\PP{Mutation-based thread interleavings exploration} 
%
To effectively explore thread interleavings, \sys mutates observed
interleaving segments recorded in \intcov. For mutating a segment,
\sys changes interleaving orders of instructions confined in each
interleaving segment.
%
In such way, \sys proactively infers what instructions' interleavings
are possible before actually running them, allowing \sys to strategize
how to test interleavings.
%
Then, \sys schedules instructions to test as many mutated interleaving
segments as possible in a single execution, quickly navigating the search space of thread interleavings.
%\sys mutates as many segments as possible and schedule generated 
%thread interleavings

% than previous approaches in which only one interleaving
% is mutated at each search step


We implement \sys across various software layers.  We extend
Syzkaller~\cite{syzkaller} to implement the core logic of the
exploration of execution paths and thread interleavings.
%
In addition, \sys incorporates an LLVM pass to transform a kernel
binary for tracking \intcov, and the QEMU/KVM-based execution engine
to enforce thread scheduling.
%
We run \sys against the latest version of the Linux kernel (ranging
from 5.19-rc2 to 6.0-rc7), and find new \totalbugs concurrency bugs all of
which exhibits harmful behaviors such as memory corruption,
task hangs, or assertion violations. We emphasize that all
concurrency bugs found by \sys were lurking in subsystems where
Syzkaller~\cite{syzkaller} has been testing for several years, which
demonstrates the usefulness of \sys in discovering concurrency bugs.
%
To quantify the efficiency of \sys, we experimentally compare \sys 
against the state-of-the-art techniques, Snowboard and KRACE, and 
demonstrate that \sys discover concurrency bugs more quickly 
(4.1$\times$ on average) than previous works.

We describe our contributions in three folds:

\begin{itemize}
\item \textbf{\Intcov:}
  %
  Based on the idea of segmentizing thread interleaving, \intcov
  encodes interleavings of multiple instructions, which provides solid
  grounds for a fuzzer to track the progress of exploring thread
  interleavings.
  %
\item \textbf{Mutation-based interleaving exploration:}
  %
  Unlike the random or simple heuristic exploration of thread
  interleavings (as in most previous approaches), the mutation-based
  interleaving exploration strategically and quickly explores the
  search space of thread interleavings.
  %
\item \textbf{Practical impact:}
  %
  We discovered new \totalbugs harmful concurrency bugs in
  recent Linux kernels in which extensive testing has been performed,
  demonstrating the practicality of \sys.
  %
\end{itemize}

%%% Local Variables:
%%% mode: latex
%%% TeX-master: "p"
%%% End:
