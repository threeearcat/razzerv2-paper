\section{Introduction}
\label{s:intro}

\dr{TODO: term consistency: interleaving order <-> interleaved instruction}

Multi-core era, many techniques are integrated in kernel to improve performance. 

The consequence of kernel concurrency bugs is disastrous. They crash
the entire system, breaking availability or causing data loss. 
Furthermore, attackers 
exploit a concurrency bugs to mount a privilege escalation attack.
A recent study~\cite{exprace} demonstrates that an user-level attacker 
reliably exploits non-deterministic concurrency bugs without performing
brute-force attacks.

Identifying kernel concurrency bugs is much more difficult than 
finding non-concurrency bugs. 
In contrast to non-concurrency bugs which can be identified by 
sequential testing of a single thread execution,
kernel concurrency bugs are typically caused by the concurrent execution 
of two or more threads.
Kernel concurrency bugs manifest only by a specific sequence of
instruction interleavings between threads. The interleavings happens 
rarely because it happens only when a specific timing condition is met.
To discover concurrency bugs, it is not practical to explore all
the possible interleavings because the number of instruction
executed by a system call in Linux is huge.
%Due to such difficulty, traditional fuzzers..

To increase efficiency of concurrency bug fuzzing, several approaches 
have been proposed. SKI, Razzer, Snowboard, Krace.\yj{what else?}
%Snowboard uses edge-coverage...

To overcome the limitation of previous approaches, this paper proposes
\sys. Key insights: 
i) new coverage metric reflecting the nature of concurrency bugs.
ii) systematic, efficient search strategy guided by the coverage metric.




We describe our contributions in three folds:

\begin{itemize}
\item interleaving segment coverage
\item speculative interleaving exploration based on explored interleaving
\item We have found \totalbugs race conditions.
\end{itemize}

%%% Local Variables:
%%% mode: latex
%%% TeX-master: "p"
%%% End:
