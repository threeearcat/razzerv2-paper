\section{Evaluation}
\label{s:eval}

To verify the effectiveness of our approach, we evaluate \sys to
answer the following questions
\begin{itemize}
\item Is \sys able to find real-world concurrency
  bugs~(\autoref{ss:realworldbugs})
\item Is \sys able to capture concurrency coverage faster than
  previous approaches~(\autoref{ss:coveragegrowth})
\item Is \XXX{} suitable to capture interesting
  behaviors~(\autoref{ss:suitability}).
\end{itemize}

\PP{Experimental setup.}
%
All of our evaluations were performed on an Intel(R) Xeon(R) CPU
E5-4655 v4 @ 2.50GHz (30MB cache) with 512GB of RAM.


\subsection{Finding Real-world Concurrency Bugs}
\label{ss:realworldbugs}

\begin{table}[t]
  \resizebox{\linewidth}{!}{
  \begin{tabular}{r l l l l}
  \toprule
    Crash ID & Kernel Version (Commit) & Subsystem & Crash Type & Crash Summary \\
  \midrule
  1 \\
  \midrule
  2 \\ 
  \midrule
  3 \\
  \midrule
  4 \\
  \midrule
  5 \\
  \midrule
  6 \\
  \midrule
  7 \\
  \midrule
  8 \\
  \midrule
  9 \\
  \midrule
  10 \\
  \midrule
  11 \\
  \midrule
  12 \\
  \midrule
  13 \\
  \midrule
  14 \\
  \midrule
  15 \\
  \midrule
  16 \\
  \midrule
  17 \\
  \midrule
  18 \\
  \midrule
  19 \\
  \midrule
  20 \\
  \midrule
  21 \\
  \bottomrule
  \end{tabular}
}

%%% Local Variables:
%%% mode: latex
%%% TeX-master: "../p.tex"
%%% End:

  \centering
  \caption{}
  \label{table:newbugs}
\end{table}

In order to prove the practicality of \sys, we ran \sys on latest
versions of the Linux kernel ranging from \XXX{} to \XXX{}.  We ran
\sys for approximately two months.

\autoref{table:newbugs} summarizes crashes found by \sys. During our
experiment, \totalbugs crashes are newly found.
%




\subsection{Coverage Growth in the Concurrency Dimension}
\label{ss:coveragegrowth}

\begin{table}[t]
  \resizebox{\linewidth}{!}{
  \begin{tabular}{l l l l l}
    \toprule
    \textbf{Bug ID} & \textbf{Subsystem} & \textbf{Crash Type} & \textbf{Reference} \\
    \midrule
    CVE-2016-8655~\cite{cve20168655} & net/packet & use-after-free access & \cite{razzer, exprace} \\
    \midrule
    CVE-2017-2636~\cite{cve20172636} & drivers/tty & double-free & \cite{razzer, exprace} \\
    \midrule
    CVE-2017-7533~\cite{cve20177533} & fs/notify & slab-out-of-bound access & \cite{exprace} \\
    \midrule
    CVE-2017-17712~\cite{cve201717712} & net/ipv4 & uninitialized access & \cite{razzer, exprace} \\
    \midrule
    CVE-2019-1999~\cite{cve20191999} & drivers/android & double-free & \cite{exprace} \\
    \midrule
    CVE-2019-2025~\cite{cve20192025} & drivers/android & use-after-free access & \cite{exprace}  \\
    \midrule
    CVE-2019-6974~\cite{cve20196974} & virt/kvm & use-after-free access & \cite{exprace} \\
    \midrule
    CVE-2019-11486~\cite{cve201911486} & drivers/tty & use-after-free access & \cite{exprace} \\
    \midrule
    69e16d01d1de~\cite{snowboardbug} & net/l2tp & NULL dereference & \cite{snowboard} \\
    \bottomrule
  \end{tabular}
}

%%% Local Variables:
%%% mode: latex
%%% TeX-master: "../"
%%% End:

  \centering
  \caption{}
  \label{table:knownbugs}
\end{table}


As the coverage growth is the important performance metric, we compare
\sys with prior works.




\subsection{Suitability of Coverage Metric}
\label{ss:suitability}


%%% Local Variables:
%%% mode: latex
%%% TeX-master: "p"
%%% End:
