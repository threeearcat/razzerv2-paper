\section{Evaluation}
\label{s:eval}

To verify the effectiveness of our approach, we evaluate \sys in
various aspects.
%
Specifically, we 1) demonstrate the practicality of \sys by providing
newly found concurrency bugs in the Linux
kernel~(\autoref{ss:realworldbugs}),
%
2) provide comprehensive performance characteristics of \sys (\eg,
throughput and overheads)~(\autoref{ss:characteristics}) and,
%
3) compare \sys against prior concurrency fuzzing
techniques~(\autoref{ss:comparison}).

\subsection{Finding Real-world Concurrency Bugs}
\label{ss:realworldbugs}

In order to demonstrate the practicality of \sys, we run \sys to
discover concurrency bugs in the latest Linux kernel.

\begin{table*}[t]
  \resizebox{\linewidth}{!}{
  \begin{tabular}{r l l l l}
  \toprule
    Crash ID & Kernel Version (Commit) & Subsystem & Crash Type & Crash Summary \\
  \midrule
  1 & XXX & drivers/misc/vmw_vmci & general protection fault & general protection fault in vmci_host_poll \\
  \midrule
  2 & XXX & net/caif & use-after-free access & use-after-free Read in cfusbl_device_notify \\
  \midrule
  3 & XXX & kernel/trace & task hung & INFO: task hung in blk_trace_remove \\
  \midrule
  4 & XXX & kernel/trace & task hung & INFO: task hung in blk_trace_setup \\
  \midrule
  5 & XXX & fs/ntfs & assertion violation & kernel BUG in ntfs_read_block \\
  \midrule
  6 & XXX & drivers/net/can/slcan & use-after-free access & KASAN: use-after-free Read in slcan_receive_buf \\
  \midrule
  7 & XXX & kernel/sched & general protection fault & general protection fault in add_wait_queue_exclusive \\
  \midrule
  8 & XXX & kernel/sched & general protection fault & general protection fault in add_wait_queue \\
  \midrule
  9 & XXX & net/netfilter & general protection fault & general protection fault in cttimeout_net_exit \\
  \midrule
  10 & XXX & arch/x86/kernel & stack-out-of-bounds access & KASAN: stack-out-of-bounds Read in unwind_next_frame\\
  \midrule
  11 & XXX & drivers/usb/core & task hung & INFO: task hung in usb_get_descriptor\\
  \midrule
  12 & XXX & net/can & warning & WARNING in isotp_tx_timer_handler \\
  \midrule
  13 & XXX & fs/btfs & assertion violation& kernel BUG in assertfail\\
  % \midrule
  % 14 & XXX & net/9p & deadlock & possible deadlock in p9_req_put \\
  % \midrule
  % 15 & XXX & fs/kernfs & use-after-free access & KASAN: use-after-free Read in __kernfs_remove\\
  % \midrule
  % 16 & XXX & net/core & warning & WARNING in __dev_queue_xmit\\
  \midrule
  14 & XXX & drivers/net/slip & use-after-free & KASAN: use-after-free Read in slip_ioctl \\
  % \midrule
  % 18 & XXX & drivers/usr/core & task hung & INFO: task hung in hub_port_init\\
  \midrule
  15 & XXX & fs/ntfs/ & slab-out-of-bound access & KASAN: slab-out-of-bounds Read in ntfs_attr_find \\
  % \midrule
  % 20 & XXX & net/ipv6 & task hung & INFO: task hung in addrconf_verify_work \\
  \midrule
  16 & XXX & sound/core/oss & use-after-free & KASAN: use-after-free Read in snd_pcm_plug_read_transfer \\
  \midrule
  17 & XXX & net/key & assertion violation & kernel BUG in pfkey_send_acquire \\
  \midrule
  18 & XXX & net/rxrpc & assertion violation & kernel BUG in rxrpc_destroy_all_connections \\
  \bottomrule
  \end{tabular}
}

%%% Local Variables:
%%% mode: latex
%%% TeX-master: "../p.tex"
%%% End:

  \centering
  \caption{List of concurrency bugs newly discovered by \sys. The
    \texttt{Recurrent} column denotes that a crash was previously
    addressed but reoccurs even after its patch is applied.}
  \label{table:newbugs}
\end{table*}

\PP{Experimental setup.}
%
To discover new concurrency bugs, we evaluate \sys on a two-socket
machine equipped with Intel(R) Xeon(R) CPU E5-2683 v4 @ 2.10GHz (40M
cache) and 512\GB of RAM.
%
This machine provides 32 total cores and 64 total threads, and runs
Ubuntu Server 20.04.4 LTS on Linux 5.4.143 as a host operating system.
%
During our experiments, we launch 32 virtual machines (VMs) where each
VM is equipped with four vCPUs and 8\GB memory.

To build a guest kernel, we use a kernel configuration used by
\texttt{Syzkaller}~\cite{syzkaller} so that \texttt{Syzkaller} and
\sys search for bugs in similar kernel modules/subsystems.
%
We run intermittently \sys for approximately two monts on latest
versions of the Linux kernel ranging from 5.19-rc2 to 6.0-rc7.


\PP{Newly found concurrency bugs.}
%
During our evaluation, \sys discovers 83 unique crash titles including
ones that \texttt{Syzkaller} also finds. Among them, \totalbugs are
newly identified as harmful concurrency bugs as summarized in
\autoref{table:newbugs}.
%
This table shows that \sys is able to find bugs across the entire
kernel from specific device drivers~(\eg, \texttt{\#1}, and
\texttt{\#6}) to various network subsystems (\eg, \texttt{\#2},
\texttt{\#16}, and \texttt{\#17}); Since \sys is not tailored to
specific subsystems, \sys entails the \textit{generality} and is
applicable to various subsystems.

\sys is also able to find not only less-harmful bugs such as warnings
(\eg, \texttt{\#12}) but also critical bugs such as memory corruptions
(\eg, \texttt{\#2}, \texttt{\#6}, and \texttt{\#14}).
%
It is worth noting that in our evaluation, unlike previous
works~\cite{snowboard, krace} that rely on data race
detectors~\cite{kcsan, tsan}, all concurrency bugs are found by
observable and harmful incidents such as kernel panics or KASAN
reports.
%
\dr{}
We believe data race detectors and our approach are complementary in
finding concurrency bugs. We discuss this in the discussion
section~(\autoref{s:discussion}).

Interestingly, we find that many of bugs was previously found and
addressed, but they reoccur possibly because of their incomplete
fixes~\cite{learningfrommistakes}.
%
In \autoref{table:newbugs}, three that are marked in the
\texttt{Recurrent} column are cases that concurrency bugs reoccur even
after their fixes are applied.
%
These cases emphasize the importance of effective testing even after
bugs are exposed and fixed accordingly.






\subsection{Performance characteristics of \sys}
\label{ss:characteristics}

In this subsection, we analyze various performance characteristics of
\sys to comprehend how our approach affects the fuzzing process.
%
\subsubsection{All-inclusive evaluation}
\label{sss:allinclusive}

We first provide performance characteristics of the whole \sys.


\PP{Coverage growth.}
%
\begin{figure}[t]
  \centering
  \includegraphics[width=\linewidth]{fig/coverage_graph-crop.pdf}
  \caption{Coverage growth of \sys under different circumstances.}
  \label{fig:eval:coverage}
\end{figure}
%
Since coverage metrics are the paramount performance metric of
fuzzing, we measure the coverage growth for both code coverage (\ie,
the number of taken branches) and interelaving coverage (\ie, the
number of observed interleaving segments) during 100 hours of fuzzing.

As shown in \autoref{fig:eval:coverage}, there is a notable difference
in scale between code coverage (a line denoted by \texttt{Branch}) and
interleaving coverage (a line denoted by \texttt{Interleaving segment}).
%
While the number of taken branches just reaches to 300K, the number of
interleaving segments is over 20M. Thus, the scale of interleaving
coverage is more than 66 times of code coverage during our evaluation.
%
This result follows the conventional belief that the search space of
thread interleaving is very large.

One thing to note about this scale difference is that storing
interleaving segment coverage consumes more memory than storing branch
coverage.
%
In our implementation, both branch coverage and interleaving segment
coverage are represented as hash tables where each element is
8-byte. Therefore, storing interleaving segment coverage consumes
about 180\MB while storing branch coverage requires 2\MB.
%
While storing interleaving segment applies the high memory pressure,
we can consider this as a traditional space-time
tradeoff~\cite{spacetimetradeoff}; we invest \textit{more memory} to
discover concurrency bugs \textit{faster}.
%
% In addition, considering each VM is equipped with 8\GB memory, it is
% endurable for the fuzzer to store interleaving segment coverage using
% 180\MB (or even using ten times of 180\MB).


\PP{Fuzzing throughput.}
%
\begin{table}[t]
  \small
  \centering
  \resizebox{0.75\linewidth}{!}{
\begin{tabular}{l l l}
  \toprule
  \sys & \texttt{Syzkaller} & \texttt{Syzkaller-memtrace} \\
  \midrule
  4.55 & 8.40 & 4.74 \\
  \bottomrule
\end{tabular}
}



% syzkaller: 30234
% syzkaller-memtrace: 17608
% c2fuzz: 16386


%%% Local Variables:
%%% mode: latex
%%% TeX-master: "../p"
%%% End:

  \caption{Fuzzing throughput (\# of exec/s) of \sys and
    \texttt{Syzkaller}. \texttt{Syzkaller-memtrace} indicates
    throughput of \texttt{Syzkaller} with memory access tracing
    enabled.}
  \label{table:throughput}
\end{table}
%
All \sys's mechanisms provide benefits in finding concurrency bugs
with a cost of additional overheads and throughput degradation.
%
To comprehend the trade-off, we measure the fuzzing throughput of \sys
and compare it with the \texttt{Syzkaller}'s throughput.
%
In order to experiment in the same environment, we measure throughput
with an empty set of seed. And because both \texttt{Syzkaller} and
\sys restart VMs after an hour of fuzzing, we measure throughput in an
hour of execution in order to eliminate noises caused by, for example,
VM rebooting or kernel crashes.



\autoref{table:throughput} shows the result. As expected, \sys shows
the lower throughput than \texttt{Syzkaller}. In particular, the
\sys's throughput is about 54\% of the \texttt{Syzkaller}'s
throughput.
%
To further understand why the \sys's throughput is degraded, we
additionally measure throughput of \texttt{Syzkaller} with memory
access tracing enabled~(\autoref{ss:instrumentation}) (but not making
use of it).
%
As shown in the \texttt{Syzkaller-memtrace} column in
\autoref{table:throughput}, it shows the throughput similar to that of
\sys; the \texttt{Syzkaller-memtrace}'s throughput is just 4.1\%
higher than the throughput of \sys.


These results indicate that the throughput of \sys is mainly degraded
by the heavy instrumentation to trace memory accesses.
%
However, as Krace~\cite{krace} states, it can be understandable as a
trade-off between throughput and input quality.
%
In the fuzzer's perspective, while tracking memory accesses has
negative impacts on throughput, it provides a higher chance for a
fuzzer to execute more interesting inputs (\ie, interesting thread
interleaving) and not to waste computing resources.
%
The effectiveness of high input quality is pronounced in
\autoref{ss:comparison}, showing \sys can discover concurrency bugs
extremely fast.




\PP{Per-input execution time}
%
\begin{table}[t]
  \centering
  \resizebox{\linewidth}{!}{
  \begin{tabular}{l l l l l l}
    \toprule
    \multirow{2}{*}{Total} & \multicolumn{2}{c}{Computation} & \multicolumn{3}{c}{Runtime} \\
    \cmidrule(lr){2-3}
    \cmidrule(lr){4-6}
     & Mutation & Coverage & Exec. & w/o XXX & w/o XXX \\
    \midrule
    267.2 & 17.2 & 8.9 & 241.1 & 198.3 & 107.6 \\
    \bottomrule
  \end{tabular}
}

% temporary
% c2fuzz
%  execute            241116156.29326048
%  mutate             17231640.620481927
%  post               8858728.732142856
% syzkaller           107640966.51956181
% syzkaller-memtrace  198337534.77521613

%%% Local Variables:
%%% mode: latex
%%% TeX-master: "../p"
%%% End:

  \caption{\dr{rewrite:} Elapsed time (ms) for executing one input.}
  \label{table:elapsedtime}
\end{table}
%
To further closely examine \sys's overheads, we measure the elapsed
time for an iteration of a fuzzing loop, and break down the elapsed
time.
%
Specifically, each iteration consists of three steps

into time taken by computation (\ie, incurred by generating a
schedule and checking new coverage) and execution of an input.
%
It is worth noting that executing an input entails two overheads for
controling thread interleaving and tracing memory accesses. To
identify how much each overhead occupies, we disable XXX and YYY, and
again measure elapsed times.
%
For each measurement, we run 10 thousands times and take an average.

\autoref{table:elapsedtime} shows the result. When executing the
input, the computational overhead is approximately 9\% of the total
elapsed time. In contrast, executing an input takes the longest time
out of the total elapsed time.
\dr{}


\subsubsection{Impact on coverage growth}
\label{sss:component}
%
Here, we present the impact of our design choices on both the
interleaving coverage growth and the code coverage growth.


\PP{Impact on interleaving exploration}
%
As the primiary purpose of this work is to effectively explore thread
interleavings, the \sys's thread scheduling
control~(\autoref{ss:scheduler}) should contribute in effectively
expanding the interleaving coverage.
%
To see how much \sys improves thread interleaving exploration, we
disable the thread scheduling control in the multi-thread fuzzing
phase of \sys, and measure the interleaving coverage.
%
The result is described in a line denotted by \texttt{Interleaivng
  segment w/o scheduling control} in \autoref{fig:eval:coverage}.
%
With the thread scheduling control disabled, \sys finds 29.1\% less
interleaving segment coverage during the same period.
%
This result indicates that our design choices significantly improve in
exploring thread interleavings.
%
Especially, \dr{}

\PP{Impact on code exploration}
%
Since \sys invests the computing power to repeatedly execute a
multi-thread input (\ie, the multi-thread fuzzing phase), we expect
that \sys might explore code coverage less than the baseline
\texttt{Syzkaller}.
%
To see the difference of the code coverage exploration in \sys and
\texttt{Syzkaller}, we measure code coverage of \texttt{Syzkaller} and
illustrate it as a line denotted by \texttt{Branch (Syzkaller)} in
\autoref{fig:eval:coverage}.
%
As a result, \sys finds 3.2\% less code coverage compared to the
baseline \texttt{Syzkaller}.
%
\dr{}






\subsection{Comparison with prior approaches}
\label{ss:comparison}

\begin{table}[t]
  \resizebox{\linewidth}{!}{
  \begin{tabular}{l l l l l}
  \end{tabular}
}

%%% Local Variables:
%%% mode: latex
%%% TeX-master: "../"
%%% End:

  \centering
  \caption{Known concurrency bugs that are studied in previous works,
    MoonShine~\cite{moonshine}, Razzer~\cite{razzer},
    ExpRace~\cite{exprace}, FUZE~\cite{fuze}, and
    Snowboard~\cite{snowboard}.}
  \label{table:knownbugs}
\end{table}

We compare \sys against various prior approaches to demonstrate the
performance improvement \sys in discovering concurrency bugs.

\PP{Bug selection}
%
\autoref{table:knownbugs} represents concurrency bugs we use for the
comparison study.
%
For fair comparisons, we select kernel concurrency bugs that are used
to evaluate previous studies on kernel concurrency bugs~\cite{exprace,
  razzer, snowboard, moonshine, fuze}.
%
Specifically, among concurrency bugs used in previous studies, we
select ones that their exploits are publicly available so that we can
make use of them for our evaluation.
%
Although the exploit of \texttt{69e16d01d1de}~\cite{snowboardbug} is
not publicly available, we successfully reproduce the concurrency bug
from the description in the Snowboard~\cite{snowboard} paper.
%
Whereas, even though Krace~\cite{krace} studies kernel concurrency
bugs (\ie, data races), we do not have access to concurrency bugs that
the authors use to evaluate Krace, and we exclude two concurrency bugs
(\ie, CVE-2019-1999~\cite{cve20191999} and
CVE-2019-2025~\cite{cve20192025}) used by ExpRace~\cite{exprace}
because \dr{XXX}.



\PP{Kernel preparation}
%
Since concurrency bugs in \autoref{table:knownbugs} are introduced,
found, and fixed at different times, it is hard to find a kernel
version that is vulnerable to all listed concurrency bugs.
%
Therefore, we inject the concurrency bugs into the Linux kernel
version v6.0-rc7 by reverting patches fixing the concurrency bugs.




\PP{Comparison methodology}
%
\begin{figure}[t]
  \caption{A multi-thread input that causes
    CVE-2017-17712~\cite{cve201717712}. The concurrency bugs may
    manifests if two system calls in bold exeucte a specific thread
    interleaving.}
  \label{fig:multithreadinput}
\end{figure}
%
Arguably, the most straightforward metric to compare fuzzing
techniques is the elapsed time until concurrency bugs are discovered.
%
However, the elapsed time heavily depends on the randomness;
\dr{because a fuzzer generates an input program very randomly, it is
  possible that one fuzzer quickly generates an input program that
  causes a concurrency bug, while another fuzzer takes very long time
  to generate the input program}.

In order to minimize the impact of the randomness and to concentrate
on the performance impact of scheduling mechanisms, we predefine a
multi-thread input as shown in \autoref{fig:multithreadinput}, and let
a fuzzer repeatedly execute the given multi-thread input without
generating new inputs nor mutating syscalls in the given input.

In addition, previous fuzzing approaches


\PP{ZZZ}
%
\begin{figure*}[t]
  \centering
  \includegraphics[width=\linewidth]{fig/comparison_graph-crop.pdf}
  \caption{\dr{WIP. Snowboard is not implemented yet} \dr{table is better?}}
  \label{fig:eval:comparison}
\end{figure*}
%

\autoref{fig:eval:comparison} shows the comparison result.
%



The difficulty of triggering a concurrency bug is different.



\dr{}
\PP{WWW}







% Single
% cve-2016-8655 O
% cve-2017-15649 O
% cve-2017-17712 X
% cve-2017-2636 X
% cve-2017-7533 X
% cve-2019-6974 X
% cve-2019-11486 X
% 69e16d01d1d O



%%% Local Variables:
%%% mode: latex
%%% TeX-master: "p"
%%% End:
