\section{Evaluation}
\label{s:eval}

To verify the effectiveness of our approach, we evaluate \sys in
various aspects. Specifically, we 1) demonstrate the practicality of
\sys by providing newly found concurrency
bugs~(\autoref{ss:realworldbugs}), 2) comprehensive \sys's performance
characteristics (\eg, throughput and
overheads)~(\autoref{ss:characteristics}), and comparison studies
against state-of-the-art concurrency fuzzing~(\autoref{ss:comparison}).

\subsection{Finding Real-world Concurrency Bugs}
\label{ss:realworldbugs}

\begin{table*}[t]
  \resizebox{\linewidth}{!}{
  \begin{tabular}{r l l l l}
  \toprule
    Crash ID & Kernel Version (Commit) & Subsystem & Crash Type & Crash Summary \\
  \midrule
  1 \\
  \midrule
  2 \\ 
  \midrule
  3 \\
  \midrule
  4 \\
  \midrule
  5 \\
  \midrule
  6 \\
  \midrule
  7 \\
  \midrule
  8 \\
  \midrule
  9 \\
  \midrule
  10 \\
  \midrule
  11 \\
  \midrule
  12 \\
  \midrule
  13 \\
  \midrule
  14 \\
  \midrule
  15 \\
  \midrule
  16 \\
  \midrule
  17 \\
  \midrule
  18 \\
  \midrule
  19 \\
  \midrule
  20 \\
  \midrule
  21 \\
  \bottomrule
  \end{tabular}
}

%%% Local Variables:
%%% mode: latex
%%% TeX-master: "../p.tex"
%%% End:

  \centering
  \caption{List of concurrency bugs newly discovered by \sys.}
  \label{table:newbugs}
\end{table*}

In order to demonstrate the practicality of \sys, we ran \sys on
latest versions of the Linux kernel ranging from \XXX{} to \XXX{}.  We
ran \sys for approximately two months.

\PP{Experimental setup.}
%
All of our evaluations were performed on an Intel(R) Xeon(R) CPU
E5-4655 v4 @ 2.50GHz (30MB cache) with 512GB of RAM.


\PP{Newly found concurrency bugs.}
%
\autoref{table:newbugs} summarizes crashes found by \sys. During our
experiment, \totalbugs crashes are newly found.
%






\subsection{Fuzzing Characteristics}
\label{ss:characteristics}

As the coverage growth is the important performance metric, we compare
\sys with prior works.




\subsection{Comparison with prior approaches}
\label{ss:comparison}

\begin{table}[t]
  \resizebox{\linewidth}{!}{
  \begin{tabular}{l l l l l}
    \toprule
    \textbf{Bug ID} & \textbf{Subsystem} & \textbf{Crash Type} & \textbf{Reference} \\
    \midrule
    CVE-2016-8655~\cite{cve20168655} & net/packet & use-after-free access & \cite{razzer, exprace} \\
    \midrule
    CVE-2017-2636~\cite{cve20172636} & drivers/tty & double-free & \cite{razzer, exprace} \\
    \midrule
    CVE-2017-7533~\cite{cve20177533} & fs/notify & slab-out-of-bound access & \cite{exprace} \\
    \midrule
    CVE-2017-17712~\cite{cve201717712} & net/ipv4 & uninitialized access & \cite{razzer, exprace} \\
    \midrule
    CVE-2019-1999~\cite{cve20191999} & drivers/android & double-free & \cite{exprace} \\
    \midrule
    CVE-2019-2025~\cite{cve20192025} & drivers/android & use-after-free access & \cite{exprace}  \\
    \midrule
    CVE-2019-6974~\cite{cve20196974} & virt/kvm & use-after-free access & \cite{exprace} \\
    \midrule
    CVE-2019-11486~\cite{cve201911486} & drivers/tty & use-after-free access & \cite{exprace} \\
    \midrule
    69e16d01d1de~\cite{snowboardbug} & net/l2tp & NULL dereference & \cite{snowboard} \\
    \bottomrule
  \end{tabular}
}

%%% Local Variables:
%%% mode: latex
%%% TeX-master: "../"
%%% End:

  \centering
  \caption{Known CVEs caused by kernel concurrency bugs.}
  \label{table:knownbugs}
\end{table}

We compare \sys against various prior approaches to provide a .

\PP{Bug selection}
%
\autoref{table:knownbugs} represents concurrency bugs we use for the
comparison study.
%
For fair comparisons, among all kernel concurrency bugs that are used
to evaluate previous studies on kernel concurrency bugs~\cite{exprace,
  razzer, snowboard, krace}, we select ones that their exploits are
publicly available such that we can make use of them for our evaluation.
%
In particular, the exploit of
\texttt{69e16d01d1de}~\cite{snowboardbug} is not publicly available,
we successfully reproduce the concurrency bug from the description in
the Snowboard~\cite{snowboard} paper.
%
Also, even though KRace~\cite{krace} studies kernel concurrency bugs
(\ie, data races), we do not have access to concurrency bugs that the
authors use to evaluate KRace.

\PP{Overall }
%







%%% Local Variables:
%%% mode: latex
%%% TeX-master: "p"
%%% End:
