\section{Evaluation}
\label{s:eval}

To verify the effectiveness of our approach, we evaluate \sys in
various aspects. Specifically, we 1) demonstrate the practicality of
\sys by providing newly found concurrency
bugs~(\autoref{ss:realworldbugs}), 2) comprehensive \sys's performance
characteristics (\eg, throughput and
overheads)~(\autoref{ss:characteristics}), and comparison studies
against state-of-the-art concurrency fuzzing~(\autoref{ss:comparison}).

\subsection{Finding Real-world Concurrency Bugs}
\label{ss:realworldbugs}

\begin{table}[t]
  \resizebox{\linewidth}{!}{
  \begin{tabular}{r l l l l}
  \toprule
    Crash ID & Kernel Version (Commit) & Subsystem & Crash Type & Crash Summary \\
  \midrule
  1 & XXX & drivers/misc/vmw_vmci & general protection fault & general protection fault in vmci_host_poll \\
  \midrule
  2 & XXX & net/caif & use-after-free access & use-after-free Read in cfusbl_device_notify \\
  \midrule
  3 & XXX & kernel/trace & task hung & INFO: task hung in blk_trace_remove \\
  \midrule
  4 & XXX & kernel/trace & task hung & INFO: task hung in blk_trace_setup \\
  \midrule
  5 & XXX & fs/ntfs & assertion violation & kernel BUG in ntfs_read_block \\
  \midrule
  6 & XXX & drivers/net/can/slcan & use-after-free access & KASAN: use-after-free Read in slcan_receive_buf \\
  \midrule
  7 & XXX & kernel/sched & general protection fault & general protection fault in add_wait_queue_exclusive \\
  \midrule
  8 & XXX & kernel/sched & general protection fault & general protection fault in add_wait_queue \\
  \midrule
  9 & XXX & net/netfilter & general protection fault & general protection fault in cttimeout_net_exit \\
  \midrule
  10 & XXX & arch/x86/kernel & stack-out-of-bounds access & KASAN: stack-out-of-bounds Read in unwind_next_frame\\
  \midrule
  11 & XXX & drivers/usb/core & task hung & INFO: task hung in usb_get_descriptor\\
  \midrule
  12 & XXX & net/can & warning & WARNING in isotp_tx_timer_handler \\
  \midrule
  13 & XXX & fs/btfs & assertion violation& kernel BUG in assertfail\\
  % \midrule
  % 14 & XXX & net/9p & deadlock & possible deadlock in p9_req_put \\
  % \midrule
  % 15 & XXX & fs/kernfs & use-after-free access & KASAN: use-after-free Read in __kernfs_remove\\
  % \midrule
  % 16 & XXX & net/core & warning & WARNING in __dev_queue_xmit\\
  \midrule
  14 & XXX & drivers/net/slip & use-after-free & KASAN: use-after-free Read in slip_ioctl \\
  % \midrule
  % 18 & XXX & drivers/usr/core & task hung & INFO: task hung in hub_port_init\\
  \midrule
  15 & XXX & fs/ntfs/ & slab-out-of-bound access & KASAN: slab-out-of-bounds Read in ntfs_attr_find \\
  % \midrule
  % 20 & XXX & net/ipv6 & task hung & INFO: task hung in addrconf_verify_work \\
  \midrule
  16 & XXX & sound/core/oss & use-after-free & KASAN: use-after-free Read in snd_pcm_plug_read_transfer \\
  \midrule
  17 & XXX & net/key & assertion violation & kernel BUG in pfkey_send_acquire \\
  \midrule
  18 & XXX & net/rxrpc & assertion violation & kernel BUG in rxrpc_destroy_all_connections \\
  \bottomrule
  \end{tabular}
}

%%% Local Variables:
%%% mode: latex
%%% TeX-master: "../p.tex"
%%% End:

  \centering
  \caption{List of concurrency bugs newly discovered by \sys.}
  \label{table:newbugs}
\end{table}

In order to demonstrate the practicality of \sys, we ran \sys on
latest versions of the Linux kernel ranging from \XXX{} to \XXX{}.  We
ran \sys for approximately two months.

\PP{Experimental setup.}
%
All of our evaluations were performed on an Intel(R) Xeon(R) CPU
E5-4655 v4 @ 2.50GHz (30MB cache) with 512GB of RAM.


\PP{Newly found concurrency bugs.}
%
\autoref{table:newbugs} summarizes crashes found by \sys. During our
experiment, \totalbugs crashes are newly found.
%






\subsection{Fuzzing Characteristics}
\label{ss:characteristics}

As the coverage growth is the important performance metric, we compare
\sys with prior works.




\subsection{Comparison with prior approaches}
\label{ss:comparison}

\begin{table}[t]
  \resizebox{\linewidth}{!}{
  \begin{tabular}{l l l l l}
  \end{tabular}
}

%%% Local Variables:
%%% mode: latex
%%% TeX-master: "../"
%%% End:

  \centering
  \caption{Known CVEs caused by kernel concurrency bugs.}
  \label{table:knownbugs}
\end{table}

In this subsection, we compare \sys against various prior approaches
to provide a .

\PP{Bug selection}
%
\autoref{table:knownbugs} represents concurrency bugs we use for the
comparison study.
%
For fair comparisons, we select concurrency bugs according to follow
criteria: 1) 
%



%%% Local Variables:
%%% mode: latex
%%% TeX-master: "p"
%%% End:
