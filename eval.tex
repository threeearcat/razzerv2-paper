\section{Evaluation}
\label{s:eval}

To verify the effectiveness of our approach, we evaluate \sys in
various aspects.
%
Specifically, we 1) demonstrate the practicality of
\sys by providing newly found concurrency
bugs~(\autoref{ss:realworldbugs}),
%
2) provide comprehensive performance characteristics of \sys (\eg,
throughput and overheads)~(\autoref{ss:characteristics}), and
%
3) compare \sys against prior concurrency fuzzing
techniques~(\autoref{ss:comparison}).

\subsection{Finding Real-world Concurrency Bugs}
\label{ss:realworldbugs}

In order to demonstrate the practicality of \sys, we 

\begin{table*}[t]
  \resizebox{\linewidth}{!}{
  \begin{tabular}{r l l l l}
  \toprule
    Crash ID & Kernel Version (Commit) & Subsystem & Crash Type & Crash Summary \\
  \midrule
  1 \\
  \midrule
  2 \\ 
  \midrule
  3 \\
  \midrule
  4 \\
  \midrule
  5 \\
  \midrule
  6 \\
  \midrule
  7 \\
  \midrule
  8 \\
  \midrule
  9 \\
  \midrule
  10 \\
  \midrule
  11 \\
  \midrule
  12 \\
  \midrule
  13 \\
  \midrule
  14 \\
  \midrule
  15 \\
  \midrule
  16 \\
  \midrule
  17 \\
  \midrule
  18 \\
  \midrule
  19 \\
  \midrule
  20 \\
  \midrule
  21 \\
  \bottomrule
  \end{tabular}
}

%%% Local Variables:
%%% mode: latex
%%% TeX-master: "../p.tex"
%%% End:

  \centering
  \caption{List of concurrency bugs newly discovered by \sys. The
    \texttt{Recurrent} column denotes that a crash was previously
    addressed but reoccurs even after its patch is applied.}
  \label{table:newbugs}
\end{table*}

\PP{Experimental setup.}
%
To discover new concurrency bugs, we evaluate \sys on a two-socket
machine equipped with Intel(R) Xeon(R) CPU E5-2683 v4 @ 2.10GHz (40MT
cache) and 512\GB of RAM.
%
This machine provides 32 total cores and 64 total threads, and runs
Ubuntu Server 20.04.4 LTS on Linux 5.4.143 as a host operating system.
%
To build a guest kernel, we use a kernel configuration used by
\texttt{Syzkaller}~\cite{syzkaller} so that \texttt{Syzkaller} and
\sys search for bugs in similar kernel modules/subsystems.
%
We run \sys for approximately two monts on latest versions of the
Linux kernel ranging from 5.19-rc2 to 6.0-rc7.


\PP{Newly found concurrency bugs.}
%
During our evaluation, \sys discovers 83 unique crash titles including
ones that \texttt{Syzkaller} also finds. Among them, \totalbugs are
newly identified as harmful concurrency bugs as summarized in
\autoref{table:newbugs}.
%
This table shows that \sys is able to find bugs across the entire
kernel from specific device drivers~(\eg, \texttt{\#1}, and
\texttt{\#6}) to various network subsystems (\eg, \texttt{\#2},
\texttt{\#16}, and \texttt{\#17}); Since \sys is not tailored to
specific subsystems, \sys entails the \textit{generality} and is
applicable to various subsystems.

\sys is also able to find not only less-harmful bugs such as warnings
(\eg, \texttt{\#12}) but also critical bugs such as memory corruptions
(\eg, \texttt{\#2}, \texttt{\#6}, and \texttt{\#14}).
%
It is worth noting that in our evaluation, unlike previous
works~\cite{snowboard, krace} that rely on data race
detectors~\cite{kcsan, tsan}, all concurrency bugs are found by
observable and harmful incidents such as kernel panics or KASAN
reports.
%
We believe data race detectors and our approach are complementary in
finding concurrency bugs. We discuss this in the discussion
section(~\autoref{s:discussion}).

Interestingly, we find that many of bugs was previously found and
addressed, but they reoccur possibly because of their incomplete
fixes~\cite{learningfrommistakes}.
%
In \autoref{table:newbugs}, three that are marked in the
\texttt{Recurrent} column are cases that reoccur long time (\eg,
hundreds days) after their fixes are applied.
%
These cases emphasize the importance of effective testing even after
bugs are exposed and fixed accordingly.






\subsection{Fuzzing Characteristics}
\label{ss:characteristics}

As the coverage growth is the important performance metric, we compare
\sys with prior works.




\subsection{Comparison with prior approaches}
\label{ss:comparison}

\begin{table}[t]
  \resizebox{\linewidth}{!}{
  \begin{tabular}{l l l l l}
    \toprule
    \textbf{Bug ID} & \textbf{Subsystem} & \textbf{Crash Type} & \textbf{Reference} \\
    \midrule
    CVE-2016-8655~\cite{cve20168655} & net/packet & use-after-free access & \cite{razzer, exprace} \\
    \midrule
    CVE-2017-2636~\cite{cve20172636} & drivers/tty & double-free & \cite{razzer, exprace} \\
    \midrule
    CVE-2017-7533~\cite{cve20177533} & fs/notify & slab-out-of-bound access & \cite{exprace} \\
    \midrule
    CVE-2017-17712~\cite{cve201717712} & net/ipv4 & uninitialized access & \cite{razzer, exprace} \\
    \midrule
    CVE-2019-1999~\cite{cve20191999} & drivers/android & double-free & \cite{exprace} \\
    \midrule
    CVE-2019-2025~\cite{cve20192025} & drivers/android & use-after-free access & \cite{exprace}  \\
    \midrule
    CVE-2019-6974~\cite{cve20196974} & virt/kvm & use-after-free access & \cite{exprace} \\
    \midrule
    CVE-2019-11486~\cite{cve201911486} & drivers/tty & use-after-free access & \cite{exprace} \\
    \midrule
    69e16d01d1de~\cite{snowboardbug} & net/l2tp & NULL dereference & \cite{snowboard} \\
    \bottomrule
  \end{tabular}
}

%%% Local Variables:
%%% mode: latex
%%% TeX-master: "../"
%%% End:

  \centering
  \caption{Known CVEs caused by kernel concurrency bugs.}
  \label{table:knownbugs}
\end{table}

We compare \sys against various prior approaches to provide a .

\PP{Bug selection}
%
\autoref{table:knownbugs} represents concurrency bugs we use for the
comparison study.
%
For fair comparisons, among all kernel concurrency bugs that are used
to evaluate previous studies on kernel concurrency bugs~\cite{exprace,
  razzer, snowboard, krace}, we select ones that their exploits are
publicly available such that we can make use of them for our evaluation.
%
In particular, the exploit of
\texttt{69e16d01d1de}~\cite{snowboardbug} is not publicly available,
we successfully reproduce the concurrency bug from the description in
the Snowboard~\cite{snowboard} paper.
%
Also, even though KRace~\cite{krace} studies kernel concurrency bugs
(\ie, data races), we do not have access to concurrency bugs that the
authors use to evaluate KRace.

\PP{Overall }
%







%%% Local Variables:
%%% mode: latex
%%% TeX-master: "p"
%%% End:
