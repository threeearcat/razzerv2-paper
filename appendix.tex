\section{Appendix}

\subsection{Virtual Machine Introspection}
\label{s:appendix:vmi}


As a hardware breakpoint does not distinguish the running context of a
kernel, if the context switch happens, a breakpoint may be hit by
another thread or an interrupt handler, making the execution out of
expectation.
%
The hypervisor recognizes a running context using \texttt{task_struct}
which holds the thread description, and the per-cpu
\texttt{preempt_count} variable indicating what context the thread is
in (\eg, a task context for running a syscall, or a hardIRQ context to
handle hardware interrupts).
%
If a breakpoint is hit by a context other than the fuzzer-controlled
thread, our hypervisor ignores it and keeps the breakpoint.



When the suspended thread already acquires a lock while the running
thread wants to hold the same lock, the whole execution cannot make a
progress, because our hypervisor forces the lock-holding thread to
suspend.
%
Therefore, our hypervisor inspects whether the running thread is going
to be blocked due to the lock contention, and if it is, our hypervisor
takes control from the running thread to the suspended thread.
%
Inspecting the lock contention is conducted by hooking lockdep
functions~\cite{lockdep} that are commonly called from synchronization
prmitives.
%
When the lockdep functions are called, our hypervisor determins
whether the running thread can make a progress through various
information such as the address of the synchronization primitive, and
operation type (\ie, lock, unlock, and trylock).


%%% Local Variables:
%%% mode: latex
%%% TeX-master: "p"
%%% End:
