\section{Related work}
\label{s:relwk}

For decades, immense efforts have been made to effectively discover
concurrency bugs. In this section, We describe prior efforts in three
categories, data race detection, controlled concurrency testing, and
kernel fuzzing.


\PP{Data race detector}
%

\cite{lkmm, linuxmemorymodel}

Although the concept of data race is widely known, the semantic

This work and data race detectors are complementary to each other.
%



\PP{Controlled concurrency testing}
%
Controlled concurrnecy testing~\cite{ski, pctalgorithm, sparsernr,
  chess, nagarakatte2012multicore, abdelrasoul2017promoting,
  cai2016radius, mukherjee2020learning, schedulebounding} (abbreviated
as CCT) introduces an idea of overriding the kernel scheduler and
testing thread interleavings methodically.
%



As far as we know, concurrency fuzzing stems from CCT techniques.
%
Specifically, Razzer~\cite{razzer}, arguably the first concurrency
fuzzer, states that Razzer is inspired by SKI~\cite{ski}, a CCT
technique aiming the kernel.




\PP{Kernel fuzzing}
%
%

IMF~\cite{imf}

Syzkaller~\cite{syzkaller},

Moonshine~\cite{moonshine}

HFL~\cite{hfl}

Healer~\cite{healer} is

Janus~\cite{janus}

Hydra~\cite{hydra}


Although we adopt Syzkaller as its sequential mutation, 




% 
To the best of our knowledge, Razzer~\cite{razzer} is the first
attempt to improve a fuzzing technique in discovering concurrency
bugs.
%
The main idea of Razzer is to guide


After Razzer, Krace~\cite{krace} asserts the necessity of a coverage
metric in the concurrency dimension. Krace defines alias coverage


MUZZ~\cite{muzz}


Snowboard~\cite{snowboard}



%%% Local Variables:
%%% mode: latex
%%% TeX-master: "p"
%%% End:
