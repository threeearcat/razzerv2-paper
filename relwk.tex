\section{Related work}
\label{s:relwk}

For decades, immense efforts have been made to effectively discover
concurrency bugs in the kernel. In this section, we describe prior
efforts in three categories, kernel fuzzing, controlled concurrency
testing, and data race detection.


\PP{Kernel fuzzing}
%
To discover vulnerabilities in the kernel, fuzzing, specifically
coverage-guided fuzzing~\cite{syzkaller, moonshine, healer, hfl, imf,
  janus, hydra, trinity, kafl, periscope, syzvegas, ksg}, has proven
to be practical and is widely used in industrial fields.
%
For example, Syzkaller~\cite{syzkaller}, a kernel fuzzer developed by
Google, has found thousands of bugs, and kernel developers very often
rely on Syzkaller to find vulnerabilities and test their
implementation.
%
To further improve the Syzkaller, and to find vulnerabilities deeper
inside the kerenl, a number of attempts have been made to incorporate
advanced techniques such as seed distillation~\cite{moonshine},
relation learning~\cite{healer}, and symbolic execution~\cite{hfl}.
%
Moreover, several works expand the input space beyond syscalls to disk
images of a file system~\cite{janus, hydra}, or the communication with
hardwares through MMIO/DMA~\cite{periscope}.
%
Although they all achieve meaningful successes, they are limited in
exploring thread interleavings, which raises a demand for discovering
concurrency bugs in the kernel.







\PP{Controlled concurrency testing}
%
Controlled concurrnecy testing~\cite{ski, pctalgorithm, sparsernr,
  chess, nagarakatte2012multicore, abdelrasoul2017promoting,
  cai2016radius, mukherjee2020learning, schedulebounding} (abbreviated
as CCT) introduces an idea of overriding the kernel scheduler and
testing thread interleavings methodically.
%
Specifically, this line of approaches aims to systematically explore
interleavings of a \textit{given} test scenario.
%
In addition, they adopt various techniques to reduce the search space
of thread interleavings such as preemption
bounding~\cite{pctalgorithm, ski}, delay
bounding~\cite{delaybounding}, and dynamic partial order
reduction~\cite{dpor1, dpor2, dpor3}.



To the best of our knowledge, concurrency fuzzing stems from CCT
techniques with an idea of merging \textit{the test case generation}
(\ie, fuzzing) and \textit{the thread interleaving exploration} (\ie,
CCT).
%
Specifically, Razzer~\cite{razzer}, arguably the first concurrency
fuzzer, states that Razzer is inspired by SKI~\cite{ski}, a CCT
technique aiming the kernel.
%
Other concurrency fuzzers~\cite{krace, muzz, snowboard, conzzer} are
also (at least indirectly) affected by CCT techniques as all of them
consists of thread scheduling control mechanisms.
%
We believe \sys takes one step further in this direction by answering
the question \textit{``what is the proper interleaving coverage
  metric''} and \textit{``how to utilize interleaving coverage''}.



\PP{Data race detection}
%

\cite{lkmm, linuxmemorymodel}

Although the concept of data race is widely known, the semantic

This work and data race detectors are complementary to each other.
%




%%% Local Variables:
%%% mode: latex
%%% TeX-master: "p"
%%% End:
