\section{Related work}
\label{s:relwk}

% For decades, immense efforts have been made to effectively discover
% concurrency bugs in the kernel.
%
In this section, we describe prior efforts such as, kernel fuzzing,
controlled concurrency testing, and data race detection.


\PP{Kernel fuzzing}
%
To discover vulnerabilities in the kernel, fuzzing, specifically
coverage-guided fuzzing~\cite{syzkaller, moonshine, healer, hfl, imf,
  janus, hydra, trinity, kafl, periscope, syzvegas, ksg}, has proven
to be practical and is widely used in industrial fields.
%
% For example, Syzkaller~\cite{syzkaller}, a kernel fuzzer developed by
% Google, has found thousands of bugs through a few years of running.
%
Then, to further improve a kernel fuzzing, a number of attempts have
been made to incorporate advanced techniques~\cite{moonshine, healer, hfl},
% such as seed distillation~\cite{moonshine}, relation
% learning~\cite{healer}, and symbolic execution~\cite{hfl}.
%
or to expand the input space beyond syscalls~\cite{janus, hydra,
  periscope}.
%
Although they all achieve meaningful successes, they are limited in
exploring thread interleavings, which raises a demand for discovering
concurrency bugs in the kernel.







\PP{Controlled concurrency testing (CCT)}
%
CCT~\cite{ski, pctalgorithm, sparsernr, chess,
  nagarakatte2012multicore, abdelrasoul2017promoting, cai2016radius,
  mukherjee2020learning, schedulebounding} introduces an idea of
overriding the kernel scheduler and methodically testing thread
interleavings of a \textit{given} input.
%
% In addition, they adopt various techniques to reduce the search space
% of thread interleavings such as
% %
% % preemption bounding~\cite{pctalgorithm, ski}, delay
% % bounding~\cite{delaybounding}, and
% %
% dynamic partial order reduction~\cite{dpor1, dpor2, dpor3}.
%
To the best of our knowledge, concurrency fuzzing stems from CCT with
an idea of merging \textit{the test case generation} (\ie, fuzzing)
and \textit{the thread interleaving exploration} (\ie, CCT).
%
Specifically, Razzer~\cite{razzer}, arguably the first kernel
concurrency fuzzer, states that it is inspired by SKI~\cite{ski}, a
CCT technique aiming the kernel.
%
We believe \sys takes one step further in this direction by adopting
interleaving segment coverage in tracking and exploring the
interleaving space.
%
% Other concurrency fuzzers~\cite{krace, muzz, snowboard, conzzer} are
% also (at least indirectly) affected by CCT techniques as all of them
% consists of thread scheduling control mechanisms.
%
% We believe \sys takes one step further in adopting CCT into fuzzing.
% by
% answering two questions, \textit{``what is the proper interleaving
%   coverage metric''} and \textit{``how to utilize interleaving
%   coverage in searching thread interleavings''}.



\PP{Data race detection}
%
A large volume of works~\cite{pacer, datacollider, hybridchecker,
  literace, helgrind, frost, prorace, tsan, kcsan, txrace} are
proposed to detect data races, a \textit{subset} of concurrency
bugs~\cite{lkmm, linuxmemorymodel}.
%
Most data race detectors monitor the execution to check the occurence
of data races via the instrumentation~\cite{tsan, kcsan, helgrind} or
the offline analysis~\cite{krace}. Thus, they are compatible with
\sys, and not competitors of \sys.
%
We believe \sys and data race detectors contribute in discovering
concurrency bugs in their own way.





%%% Local Variables:
%%% mode: latex
%%% TeX-master: "p"
%%% End:
